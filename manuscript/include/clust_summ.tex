
% Table created by stargazer v.5.2.2 by Marek Hlavac, Harvard University. E-mail: hlavac at fas.harvard.edu
% Date and time: Fri, Nov 08, 2019 - 11:40:45
\begin{table}[!htbp] \centering 
  \caption{Description of the biogeographical clusters (C1-C5) to which each plot in the study was assigned. Indicator species were generated using Dufrene-Legendre indicator species analysis \citep{Dufrene1997} implemented with \texttt{indval()} from the \texttt{labdsv} R package \citep{labdsv}. Numeric values are means $\pm$ 1 standard deviation.} 
  \label{clust_summ} 
\begin{tabular}{@{\extracolsep{0pt}} clccc} 
\\[-1.8ex]\hline 
\hline \\[-1.8ex] 
{Cluster} & {Indicator species} & {Species richness} & {Stems ha\textsuperscript{-1}} & {AGB (t ha\textsuperscript{-1})} \\
\hline \\[-1.8ex] 
C1 & \begin{tabular}[l]{@{}l@{}l@{}} \textit{Diplorhynchus condylocarpon} \\ \textit{Combretum }spp. \\ \textit{Pseudolachnostylis maprouneifolia} \end{tabular} & 7$\pm$4.7 & 276$\pm$345.2 & 35.3$\pm$37.03 \\ 
\hline
C2 & \begin{tabular}[l]{@{}l@{}l@{}} \textit{Julbernardia paniculata} \\ \textit{Isoberlinia angolensis} \\ \textit{Albizia antunesiana} \end{tabular} & 8$\pm$4.3 & 274$\pm$245.9 & 46.8$\pm$35.05 \\ 
\hline
C3 & \begin{tabular}[l]{@{}l@{}l@{}} \textit{Burkea africana} \\ \textit{Pterocarpus angolensis} \\ \textit{Baikiaea plurijuga} \end{tabular} & 4$\pm$2.3 & 274$\pm$536.5 & 36.3$\pm$27.86 \\ 
\hline
C4 & \begin{tabular}[l]{@{}l@{}l@{}} \textit{Baikiaea plurijuga} \\ \textit{Terminalia randii} \\ \textit{Albizia amara} \end{tabular} & 10$\pm$5.9 & 447$\pm$386.3 & 47$\pm$45.41 \\ 
\hline
C5 & \begin{tabular}[l]{@{}l@{}l@{}} \textit{Colophospermum mopane} \\ \textit{Pseudolachnostylis maprouneifolia} \end{tabular} & 4$\pm$2.9 & 322$\pm$390.1 & 43.8$\pm$34.45 \\ 
\hline \\[-1.8ex] 
\end{tabular} 
\end{table} 
